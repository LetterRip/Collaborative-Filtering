% TEX encoding = UTF-8 Unicode
% Compiled by XeLaTeX.

% Compiled at least 2 times, for generating contents and figure reference.
% if documents are needed, using BibTeX first after aux file generated.

\documentclass[a4paper,12pt]{ctexart}
\usepackage{xeCJK}
\usepackage{setspace}				% set line space
%\usepackage{amsmath}				% Math package, like align
%\usepackage{bbm}					% use indicator function in math


\title{推荐系统的协同过滤算法实现与浅析}
%\author{}
%\date{\today}

\begin{document}
\begin{spacing}{1.35}

\maketitle
\tableofcontents
\newpage

\section{摘要}

\subsection{项目描述}

个性化推荐系统基于用户的历史兴趣和商品的特性,向用户推荐合适的信息或商品,其目前在互联网领域,尤其是电子商务、广告业务等方面,具有非常广泛的应用。推荐系统的进步,会更加迎合用户的需求,为产品赢得更好的口碑,创造更多的收益,形成一个良性的循环。因而,对推荐系统的研究在算法和实践中不断发展,不断进步。

本项目选择推荐系统为主题,基于(user, item, rating)数据集,预测没有评价的(user, item),以均方根误差RMSE为评价指标。

由于这门课是以算法为核心的课程,因而本项目更加注重算法的具体内容和细节。在项目中,所有核心代码均自己编写,没有调用任何外部算法模块。

在baseline的基础上,我实现了最基本的user-user和item-item协同过滤算法,以及基于baseline的协同过滤算法,验证了item-item相比user-user能获得更好的效果。同时,在基本的协同过滤算法上,加上了bias、TopK等优化,进一步提升了模型效果。此外,研究了TopK算法中,K的取值对模型效果的影响。最后,尝试融合了不同的算法并调参,获得了更好的融合模型。

在项目过程中,在计算统计量、相似度、预测等处,遇到了一些算法细节问题,并进行了合适的处理。同时,自己重新组织了代码,使其具有更好的模块性,运行更加流水线化。

此外,对于数据集采用了5-fold 交叉验证,以交叉验证后的均值RMSE作为更稳定的评价指标。

\subsection{数据集}

原先,我采用的是Netflix-Prize 数据集。标准的Netflix-Prize数据集包含了480189个user,和17770个item,以及总计约1亿的ratings.

对于协同过滤方法来说,该数据集产生的评分矩阵达$480189 \times 17770$维,在该矩阵上的统计已经要耗时数十秒,对该矩阵进行更细粒度的计算则会更慢。

因而,我换用了一个规模更小的数据集MovieLens, 其数据方式与Netflix-Prize基本

\subsection{评价指标}

训练集和测试集规模比为$4:1$。

评价指标选择均方根误差,其定义如下:

\[RMSE = \sqrt{\sum_{i = 1}^{N}(r_{xi} - \hat{r}_{xi})^{2}}\]
其中,$\hat{r}_{xi}$为用户$x$对项目$i$的预测打分,$r_{xi}$为用户$x$对项目$i$的实际打分。

在模型改进时,采用了交叉验证的方法,最后选择每组数据集的RMSE均值作为预测方法的最终RMSE值:(k为交叉验证的组数)

\[\overline{RMSE} = \frac{1}{k}\sum_{i=1}^{k}RMSE_{i}\]

\section{模型介绍与分析}

\subsection{统计指标及baseline}

\subsection{相似度矩阵与协同过滤}

\subsection{结合baseline的协同过滤}

\subsection{TopK协同过滤与K值的选择}

\subsection{模型融合与融合参数}

\section{进一步工作}

\section{总结}

本项目所有源代码已在GitHub开源,地址为{https://github.com/irmowan/MovieLens} .

\renewcommand\refname{参考文献}
\bibliographystyle{plain}
\bibliography{Report}

\end{spacing}
\end{document}
